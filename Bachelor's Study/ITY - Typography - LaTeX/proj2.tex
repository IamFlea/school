%Jmeno: Petr Dvoracek
%E-mail: xdvora0n@stud.fit.vutbr.cz
%Projekt c. 2
%
%Text dokumentu
%--------------

\documentclass[11pt,a4paper,twocolumn]{article}
\usepackage[left=1.5cm,text={18cm, 25cm},top=2.5cm]{geometry}
\usepackage[czech]{babel}
\usepackage[IL2]{fontenc}
\usepackage[utf8]{inputenc}
\usepackage{amsmath}
\usepackage{amsthm}
\usepackage{amsfonts} 
\newcommand{\myuv}[1]{\quotedblbase #1\textquotedblleft}
\hyphenation{re-pre-zen-tu-je ma-te-ma-tic-kou}

\begin{document}
\begin{titlepage}
\begin{center}
{\Large \textsc{Fakulta informačních technologií \\ Vysoké učení technické v~Brně}}\\
%\end{center}
\vspace{\stretch{0.382}}
%\begin{center}
%\vfill
{\LARGE Typografie a publikování -- 2. projekt}\\
\medskip
{\Huge Sazba dokumentů a matematických výrazů}
%\vfill
\vspace{\stretch{0.618}}
\end{center}

{\Large { \today \hfill Petr Dvořáček}}\\
\end{titlepage}

\section*{Úvod} 
V~této úloze si vyzkoušíme sazbu titulní strany, matematických vzorců, prostředí a dalších textových struktur obvyklých pro technicky zaměřené texty (například rovnice (\ref{rovnice1}) nebo definice \ref{definice1.1} na straně \pageref{definice1.1}).

Na titulní straně je využito sázení nadpisu podle optického středu s~využitím \emph{zlatého řezu}. Tento postup byl probírán na přednášce.


\section{Matematický text}
Nejprve se podíváme na sázení matematických symbolů a výrazů v~plynulém textu. Pro množinu $V$ označuje $\mbox{card}(V)$ kardinalitu $V$. Pro množinu $V$ reprezentuje $V^*$ volný monoid generovaný množinou $V$ s~operací konkatenace. Prvek identity ve volném monoidu $V^*$ značíme symbolem $\varepsilon$.
Nechť $V^+ = V^* - \{\varepsilon\}$. Algebraicky je tedy $V^+$ volná pologrupa generovaná množinou $V$ s~operací konkatenace. Konečnou neprázdnou množinu $V$ nazvěme \emph{abeceda}. Pro $w \in V^*$ označuje $|w|$ délku řetězce $w$. Pro $W \subseteq V$ označuje $\mbox{occur}(w,W)$ počet výskytů symbolů z~$W$ v~řetězci $w$ a $\mbox{sym}(w,i)$ určuje $i$-tý symbol řetězce $w$; například $\mbox{sym}(abcd,3)=c$.


Nyní zkusíme sazbu definic a vět s~využitím balíku \texttt{amsthm}.


\theoremstyle{definition}
\newtheorem{definice}{Definice}[section]

\begin{definice} \label{definice1.1}  \emph{Bezkontextová gramatika} je čtveřice $G=(V,T,P,S)$, kde $V$ je totální abeceda, $T \subseteq V$ je abeceda terminálů, $S \in (V-T)$ je startující symbol a $P$ je konečná množina \emph{pravidel} tvaru $q \colon A \rightarrow \alpha$, kde $A \in (V - T) $, $\alpha \in V^*$ a $q$ je návěští tohoto pravidla. Nechť $N = V - T$ značí abecedu neterminálů.
Pokud $q \colon A \rightarrow \alpha \in P$, $\gamma, \delta \in V^*$, $G$ provádí derivační krok z~$\gamma A \delta$ do $\gamma \alpha \delta$ podle pravidla $q\colon A \rightarrow \alpha$, symbolicky píšeme $\gamma A \delta \Rightarrow \gamma \alpha \delta\ [q\colon A \rightarrow \alpha]$ nebo zjednodušeně $\gamma A \delta \Rightarrow \gamma \alpha \delta$. Standardním způsobem definujeme $\Rightarrow^m$, kde $m \geq 0 $. Dále definujeme tranzitivní uzávěr $\Rightarrow^+$ a tranzitivně-reflexivní uzávěr $\Rightarrow^*$.
\end{definice}


Algoritmus můžeme uvádět podobně jako definice textově, nebo využít pseudokódu vysázeného ve vhodném prostředí (například \texttt{algorithm2e}).

\theoremstyle{definition}
\newtheorem{algoritmus}[definice]{Algoritmus}

\begin{algoritmus}  Ověření bezkontextovosti gramatiky. Mějme gramatiku $G = (N, T, P, S)$.
\begin{enumerate}
\item \label{krok} Pro každé pravidlo $p \in P$ proveď test, zda $p$ na levé straně obsahuje právě jeden symbol z~$N$.
\item Pokud všechna pravidla splňují podmínku z~kroku \ref{krok}, tak je gramatika $G$ bezkontextová.
\end{enumerate}
\end{algoritmus} 

\begin{definice}  \emph{Jazyk} definovaný gramatikou $G$ definujeme jako $L(G) = \{w \in T^*\ |\ S \Rightarrow^* w \}$.
\end{definice}

\subsection{Podsekce obsahující větu}

\begin{definice} 
Nechť $L$ je libovolný jazyk. $L$ je \emph{bezkontextový jazyk}, když a jen když $L = L(G)$, kde $G$ je libovolná bezkontextová gramatika.
\end{definice}

\begin{definice}
Množinu $\mathcal{L}_{CF} = \{L | L$ je bezkontextový jazyk$\}$ nazýváme \emph{třídou bezkontextových jazyků}.
\end{definice}

\theoremstyle{definition}
\newtheorem{lemma}{Věta}

\begin{lemma}  \emph{Nechť} $L_{abc} = \{ a^n b^n c^n | n \geq 0 \}$. \emph{Platí, že} $L_{abc} \not\in \mathcal{L}_{CF}$. \label{veta} \end{lemma}

\begin{proof}  Důkaz se provede pomocí Pumping lemma pro bezkontextové jazyky, kdy ukážeme, že není možné, aby platilo, což bude implikovat pravdivost věty \ref{veta}.\end{proof}

\section{Rovnice a odkazy}

Složitější matematické formulace sázíme mimo plynulý text. Lze umístit několik výrazů na jeden řádek, ale pak je třeba tyto vhodně oddělit, například příkazem \verb|\quad|. 

$$\sqrt[x^2]{y^3_0} \quad \mathbb{N} = \{1,2,3,\ldots\} \quad x^{y^y} \neq x^{yy} \quad z_{i_j} \not\equiv {z_i}_j$$

V~rovnici (\ref{rovnice1}) jsou využity tři typy závorek s~různou explicitně definovanou velikostí.

\begin{eqnarray}
x & = & -\bigg\{\Big[\big(a + b\big)^c * d\Big] + 1\bigg\}\label{rovnice1} \\
s & = & \sqrt{\frac{1}{n}\sum\limits^r_{i=1} p_i (x_i - x)^2 \nonumber}
\end{eqnarray}

V~této větě vidíme, jak vypadá implicitní vysázení limity $\lim_{n \to \infty}f(n)$ v~normálním odstavci textu. Podobně je to i s~dalšími symboly jako $\sum_1^n$ či $\bigcup_{A\in \mathcal{B}}$. V~případě vzorce $\lim\limits_{x \to 0}\frac{\sin x}{x}=1$ jsme si vynutili méně úspornou sazbu příkazem \verb|\limits|.

\begin{eqnarray}
\int\limits_a^b f(x)\, \mathrm{d}x &=& -\int_b^a f(x)\, \mathrm{d}x  \\
\overline{\overline{A} \wedge \overline{B}} &=& \overline{\overline{A \vee B}}
\end{eqnarray}

\section{Složené zlomky}
Při sázení složených zlomků dochází ke zmenšování použitého písma v~čitateli a jmenovateli. Toto chování není vždy žádoucí, protože některé zlomky potom mohou být obtížně čitelné. 

V~těchto případech je možné ručně nastavit standardní stupeň písma v~podvýrazech pomocí \verb|\displaystyle| u~vysázených vzorců nebo pomocí \verb|\textstyle| u~vzorců, které jsou součástí textu. Srovnejte:
$$\frac{\frac{a^2}{x+y}-\frac{\frac{a}{b}}{x-y}} {\frac{a+b}{a-b} - 1} \quad \frac{\displaystyle \frac{\displaystyle a^2}{\displaystyle x+y}-\frac{\displaystyle \frac{\displaystyle a}{\displaystyle b}}{\displaystyle x-y}} {\displaystyle \frac{\displaystyle a+b}{\displaystyle a-b} - 1}
$$

Tento postup lze použít nejen u~zlomků.
$$\prod\limits^{m-1}_{i=0} (n-i) = \underbrace{n(n-1)(n-2)\dots(n-m+1)}_
{\displaystyle m \mbox{ je počet činitelů}} $$
\section{Matice}

Pro sázení matic se velmi často používá prostředí \texttt{array} a závorky  (\verb|\left|, \verb|\right|). 

$$\left( \begin{array}{cc}
a + b & a - b \\
\widetilde{c + d} & \tilde{b} \\
\vec{a} & \underleftrightarrow{AC} \\
\alpha & \aleph\\
\end{array} \right)$$

$$\mathbf{A}=\left\|\begin{array}{cccc}
a_{11} & a_{12} & \ldots & a_{1n} \\
a_{21} & a_{22} & \ldots & a_{2n} \\
\vdots & \vdots & \ddots & \vdots \\
a_{m1} & a_{m2} & \ldots & a_{mn}
\end{array}\right\|$$


$$\left|\begin{array}{cc}
k & l \\ 
m & n
\end{array}\right| = kn -lm$$

Prostředí \texttt{array} lze úspěšně využít i jinde.

$$\binom{n}{k} = \left\{ 
\begin{array}{l l}
0 & \quad \mbox{pro $k < 0$ nebo $k > n$}\\
\frac{n!}{k!(n-k)!} & \quad \mbox{pro $0 \leq k \leq n$}\\
\end{array} \right. $$

\section{Závěrem}
V~případě, že budete potřebovat vyjádřit matematickou konstrukci nebo symbol a nebude se Vám dařit jej nalézt v~samotném \LaTeX u, doporučuji prostudovat možnosti balíku maker \AmS-\LaTeX.
Analogická poučka platí obecně pro jakoukoli konstrukci v~\TeX u.

\end{document}


%Základní informace o dokumentu
%------------------------------
%rozměry stránky: A4
%rozměry textové oblasti: 18x25cm
%mezera vlevo: 1.5cm
%mezera nahoře: 2.5cm
%font: standardní 11pt (vzorový dokument používá kódování fontů IL2)
%
%Poznámky: 
%- Pomlčky a spojovníky jsou v tomto textu zadány znakem -. V dokumentu musí %být správná šířka podle kontextu.
%- V tomto textu jsou použity tyto "uvozovky". Ve výsledném dokumentu musí být uvozovky podle zadaného PDF.
%- Vzorový dokument byl vysázen LaTeXem na školním serveru merlin těmito nástroji:
%  latex
%  dvips -t a4
%  ps2pdf
%  
%  Při použití pdflatexu je výsledný soubor vizuálně totožný, ale je dvakrát větší.
%
%- Vzorce a některé další elementy jsou v tomto textu nahrazeny třemi tečkami.


